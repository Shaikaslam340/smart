
\documentclass[10pt]{beamer}

\usetheme[progressbar=frametitle]{metropolis}
\usepackage{appendixnumberbeamer}
\usepackage[numbers,sort&compress]{natbib}
\bibliographystyle{plainnat}

\usepackage{booktabs}
\usepackage[scale=2]{ccicons}

\usepackage{xspace}
\newcommand{\themename}{\textbf{\textsc{metropolis}}\xspace}

\title{Smart Shelter using IoT}
 \subtitle{EYIC}
% \date{\today}
\date{}
\author{Institute : Sri Venkateswara College of Engineering}
\institute{\textbf{TEAM}\\S.AFZAL\\S.ASLAM\\S.SUBRAMANYAM\\S.SRUJANA\\\\\textbf{MENTOR}\\P.M.D.ALI KHAN}


\begin{document}


\maketitle


\begin{frame}[fragile]{Introduction}
After a brisk research about the present products that are available in the market to serve the needs, there is
no such product that is atleast nearer to similarity. Normal umbrellas serve as shield but, what if it is
forgotten, also, needs to be hold. On a similar note there are no air purification systems that are installed in
alleys. Mainly it works relying on solar power and there is no product that uses solar energy for operating
the umbrella. While coming to lighting street vendors prefer to install their stalls under street lights,
compromising on their comfort zones or high marketing areas. Thus, a product likes we proposed has not yet
come into practice or not even thought of. As, it serves the necessities, which can be operated both manually
and automatically. Entirely this is a feasible product as it is portable, low cost and ultimately is a
combination of three utilities.
\end{frame}
\begin{frame}[fragile]{Problem Statement}

  Every day we are observing that air pollution has been rapidly increasing in India due to exponential
multiplicity of vehicles. As per recent study after the festival of lights i.e., Diwali, in Delhi the air quality
dropped to hazardous levels. The motivation for this product arises from that results of sensex. Also, during
sudden drastic changes in climatic conditions most of the streets are modernized in such a way that,
providing shelter is not taken care off. This product is also helpful for open restaurants and street vendors as,
they are losing customers due to the lack of providing shelter for them during rainy season. So, we came
with the idea of providing shelter, lights and air purification as a product which can be operated in both
manual and automatic modes using mobile application. We use solar power to activate the shelter.
\end{frame}
\begin{frame}[fragile]{Requirements}

Raspberry pi 3\\
Batteries\\
Solar panel\\
Lamp with bulb\\
Tower blower\\
Activated carbon and hepa filters\\
Aluminium Pipe and rods\\
Solar power controller\\
Linear Actuator\\
Dc motors\\
\end{frame}
\begin{frame}[fragile]{Feasibility}
Providing shelter, light and air purification is the central agenda of this product. The umbrella helps people
protect themselves from extreme climatic conditions that suddenly appear. If it drizzles unexpectedly people
can save themselves by standing under the umbrella. In the case of bright and heavy sun rays also people can
be provided shade. The working of the umbrella involves solar energy. Also, an air purifier is being provided
inbuilt so as the refine the filthy air surrounded the cover. The umbrella also contains a light fabricated
within itself. The umbrella is a portable and weightless product which can be implanted anytime and
anywhere with ease. Considering the fact of maintenance of the umbrella is quite an easy task as they are
waterproof and dust proof by nature. And also, the material used while designing the umbrella is of good
quality so the problem of wear and tear doesn't occur. This umbrella can be implemented in many places
such as street food stalls, shops, parks, busy alleys, etc
\end{frame}
\section{Thanking you}
\end{document}
